\documentclass{article}

% Imports
\usepackage{fancyhdr}
\usepackage{mathtools}
\usepackage{extramarks}
\usepackage{amsmath}
\usepackage{amsthm}
\usepackage{amssymb}
\usepackage{amsfonts}
\usepackage{enumitem}


% Macros
% Homework Macros
\newcommand{\hmwkTitle}{Homework 1}
\newcommand{\hmwkClass}{Class Title}
\newcommand{\hmwkClassTime}{Section 001}
\newcommand{\hmwkClassInstructor}{Professor's Name -}
\newcommand{\hmwkAuthorName}{Student's Name}
\newcommand{\hmwkUNID}{Student ID}
\newcommand{\tabHere}{\par\setlength\parindent{24pt}}

% Basic Document Settings
\topmargin		=-0.45in
\evensidemargin	=0in
\oddsidemargin	=0in
\textwidth		=6.5in
\textheight		=9.0in
\headsep		=0.25in

\linespread{1.1}

\pagestyle{fancy}
%\lhead{\hmwkAuthorName}
\lhead{\hmwkAuthorName}
\rhead{\hmwkClass\ (\hmwkClassInstructor\ \hmwkClassTime): \hmwkTitle}
\lfoot{\lastxmark}
\cfoot{\thepage}

\renewcommand\headrulewidth{0.4pt}
\renewcommand\footrulewidth{0.4pt}

\setlength\parindent{0pt}

% Front Page
% Title Page
\title{
    \vspace{2in}
    \textmd{\textbf{\hmwkClass:\ \hmwkTitle}}\\
    \vspace{0.1in}\large{\textit{\hmwkClassInstructor\ \hmwkClassTime}}
    \vspace{3in}    
    \author{\textbf{\hmwkAuthorName}}
    \date{\today}
}


\begin{document}
	\maketitle
	\pagebreak

\subsection*{Problem 1. (Question 1, Section 7.1)}
(a.) $2x+y = (2x_1, 2x_2, 2x_3) + (y_1, y_2, y_3) = (1,3,5)$\\
(b.) $x\cdot y = -1 + 0 + 2 = 1$\\
(c.) $||x|| = \sqrt{x\cdot x} = \sqrt{1+0+4} = \sqrt{5}$,  $||y|| = \sqrt{y\cdot y} = \sqrt{1+9+1} = \sqrt{11}$\\
(d.) The cosine of the angle is given by 
\[
\frac{u \cdot v}{||u||\hspace{.1cm} ||v||} = \cos \theta
\]
Therefore the cosine of the angle between $x$ and $y$ is
\[
\frac{x \cdot y}{||x||\hspace{.1cm} ||y||} = \frac{1}{\sqrt{5}\sqrt{11}} = \frac{1}{\sqrt{55}}\approx 0.13484
\]\\
(e.) The distance is given by the metric $||x-y||$ giving us 
\[
\sqrt{(x-y) \cdot (x-y)} = \sqrt{(2,-3,1) \cdot (2,-3,1)} = \sqrt{14}
\]

\subsection*{Problem 2. (Question 6, Section 7.1)}
Let $u$ and $v$ be vectors in $\mathbb{R}^d$ where $v=au$ and $a\in\mathbb{R}$ and $a\neq0$ (trivial case). Then by the Cuachy Schwartz inequality we know that
\[
|u\cdot v| \leq ||u||\hspace{.1cm}||v||
\]
Substituting values for $v$ we get
\begin{align*}
	|u\cdot v|			 &\leq	 ||u||\hspace{.1cm}||v||  \\
	|u\cdot au|		   &\leq   ||u||\hspace{.1cm}||au||  \\
	|a(u\cdot u)|	  &\leq	  ||u||\hspace{.1cm}||au||  &\text{ by 7.1.4(c)}\\
	|a||u\cdot u|	&\leq	|a|\hspace{.1cm}||u||\hspace{.1cm}||u||  &\text{ by 7.1.10(b)}\\
	|u\cdot u|	&\leq	\sqrt{u\cdot u}\sqrt{u \cdot u} \\
	|u\cdot u|	&\leq	(u\cdot u)
\end{align*}
By by 7.1.4(d) it the inner product must always be positive, which means that 
\[
|u\cdot u|	\leq	(u\cdot u) \implies |u\cdot u|	= (u\cdot u)
\]
Therefore if $v$ is a scalar multiple of $u$ it must be an equality.

\subsection*{Problem 3. Question 8, Section 7.1}
The prove that $||\cdot ||_\infty$ is a norm of $\mathbb{R}^d$ we must show that the following three conditions hold:
\begin{itemize}
	\item[(1)] $||x+y|| \leq ||x|| + ||y||$
	\item[(2)] $||ax||  = |a|||x||$
	\item[(3)] $ ||x|| = 0 \implies x=0$
\end{itemize}
\textbf{Condition 1:}\\
Let $x\in\mathbb{R}^d$ and $y\in\mathbb{R}^d$. Also, let $x_m = \max\{|x_1|, \dotsm, |x_d|\}$ and let $y_m = \max\{|y_1|, \dotsm, |y_d|\}$. By the triangle inequality we know that
\[
|x_m + y_m|  \leq |x_m| + |y_m| \implies ||x+y||_\infty \leq ||x||_\infty + ||y||_\infty
\]
Therefore it satisfies condition 1.
\pagebreak
\\\\\textbf{Condition 2:} \\
Distributing out the absolute value we see that
\[
|ax_m| = |a||x_m| \implies |a|||x||_\infty
\]
There it satisfies condition 2.
\\\\\textbf{Condition 3:}\\
If $||x||_\infty = 0$, then $\max\{ |x_1, \dotsm, x_d\}] = 0$. Since the maximum value of any $x_j \leq x_m$ and $x_m=0$, then all $x_j=0.$. Since this forces all $x_j =0$, then $x=(0, \dotsm, 0)$ for every entry in the vector so $x=0$
\\\\
As these three conditions are satisfied, we then know that $||\cdot||_\infty$ is a norm on $\mathbb{R}^d$.

\subsection*{Problem 4 (Question 12, Section 7.1)}
Note that
\[
\sum_{k=1}^{\infty} |x_ky_k| \leq\sum_{k=1}^{\infty} |x_ky_k|^2
\]
Then summing up to $n$, we can see by Cauchy Schwartz that
\[
\sum_{k=1}^{n} x_k^2 y_k^2  \leq \sum_{k=1}^{n} x_k^2\sum_{k=1}^{n} y_k^2
\]
Taking the limit gives us $n\to\infty$
\[
\sum_{k=1}^{\infty} |x_ky_k| \leq \sum_{k=1}^{\infty} x_k^2 y_k^2  \leq \sum_{k=1}^{\infty} x_k^2\sum_{k=1}^{\infty} y_k^2 < \infty \implies \sum_{k=1}^{\infty} |x_ky_k| < \infty
\]
\subsection*{Problem 5 (Question 1, Section 7.2)}
We have that $x_n - x = \big(\frac{n}{1+n}-1, \frac{1-n}{n}+1\big)$. Then
\[
||x_n - x|| = \bigg|\bigg|\bigg(\frac{n}{1+n}-1, \frac{1-n}{n}+1\bigg)\bigg|\bigg| = \sqrt{\bigg(\frac{n}{1+n}-1\bigg) ^2  +  \bigg(\frac{1-n}{n}+1\bigg)^2} = \sqrt{\frac{1}{(n+1)^2} + \frac{1}{n^2}}
\]
Then by approximation
\[
\sqrt{\frac{1}{(n+1)^2} + \frac{1}{n^2}} \leq \sqrt{\frac{2}{n^2}} = \frac{\sqrt{2}}{n}
\]
Set $N = \frac{\sqrt{2}}{\epsilon}$, then for all $\epsilon >0$, if $n>N$ 
\[
||x_n - x|| \leq \frac{\sqrt{2}}{n} \leq \frac{\sqrt{2}}{N} = \epsilon 
\]
Showing that $(1,1)$ is indeed the limit.
\subsection*{Problem 6. (Question 2, Section 7.2)}
The sequence $\{x_n\}$ should converge. By Theorem 7.2.13  the sequence $\{x_n\}$ converges if and only if each component of $\{x_n\}$ converges to it's corresponding component of $x$. Let $x=(\frac{1}{3}, 1)$. If the limit for each component can be found then by 7.2.13 the vector should also converge. 

From 3210, we know that
\[
\frac{n^2 + n -1}{3n^2+} \to \frac{1}{3} \hspace{.5cm}\text{ and }\hspace{.5cm} \frac{n-1}{n+1} \to 1
\]
Therefore $x_n \to x$ by Theorem 7.2.13.

\subsection*{Problem 7. (Question 12, Section 7.2)}
To prove something is a metric we must show that the following conditions are satisfied:
\begin{itemize}
	\item[(1)] $d(x,y) = d(y,x)$
	\item[(2)] $d(x,y) = 0 \text{ iff } x=y$
	\item[(3)] $d(x,z) \leq d(x,y) + d(y,z)$
\end{itemize}

From the problem description, condition (2) is given. Therefore we only need to show condition (1) and 3 are satisfied. 
\\\\
(1) If $x\neq y$ then $y\neq x \implies d(y,x) = 1 = d(x,y) \implies d(x,y) = d(y,x)$ \\\\
(3) By cases, we can see that
\begin{itemize}
	\item if $x=y$         and $y=z 		\implies d(x,z) \leq d(x,y) + d(y,z) \implies 0 \leq 0$
	\item if $x\neq y$ and $y=z			\implies d(x,z) \leq d(x,y) + d(y,z) \implies 1 \leq 1$
	\item if $x= y$		   and $y\neq z	\implies d(x,z) \leq d(x,y) + d(y,z) \implies 1 \leq 1$
	\item if $x\neq y$ and $y\neq z$ and $x \neq  z\implies d(x,z) \leq d(x,y) + d(y,z) \implies 1\leq 2$ 
	\item if $x\neq y$ and $y\neq z$ and $x =  z\implies d(x,z) \leq d(x,y) + d(y,z) \implies 0\leq 2$ 
\end{itemize}
Since these are all true, then condition (3) is satisfied. Thus, since al three conditions are satisfied, we know that this is a metric on $\mathbb{R}$.

\subsection*{Problem 8. (Question 17, Section 7.2)}
Let $X$ be the set of rooms, and $d(x,y)$ be the shortest possible path from room $x$ to $y$. If we take the shortest possible path from room $x$ to $y$, then we can simply retrace our steps from room $y$ to $x$ to follow the exact same path. Therefore $d(x,y) = d(y,x)$
\\\\
If we are in room $x$ which also happens to be the same room as $y$, then the shortest path from $x$ to $y$ is to not move, since we are already in the room. Therefore $d(x,y)=0$ if $x=y$.
\\\\
Finally, If we start at room $x$ and plan to go to room $z$, but stop at at least one other room $y$ before arriving at $z$, then the distance it takes it go from room $x$ to $z$ if $y$ is along the way is the same as just going to room $z$ directly. (Assuming that "going" to a room doesn't take any extra time). However, if the room $y$ is not along the way, then we will have to spend some more time going to $y$, and then going back to room $z$, so overall the distance we walked was more than if had just went to room $z$ directly. Therefore, $d(x,z) \leq d(x,y) + d(y,z)$
\\\\
Since these three conditions are true in this situation, then our shortest possible path is a metric for the rooms in the building. 

\end{document}


